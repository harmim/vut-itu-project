\documentclass[10pt, hyperref={unicode}]{beamer}

\usepackage[czech]{babel}
\usepackage[utf8]{inputenc}
\usepackage{times}

\usetheme[progressbar=frametitle]{metropolis}
\setbeamertemplate{footline}[frame number]

\title{Tvorba uživatelských rozhraní}
\subtitle{
	\texorpdfstring{
		Projekt\,--\,TODO list pro Buddy členy organizace ESN\\
		Číslo projektu: 100\\
		Číslo a~název týmu: 41. Tým xharmi00
		}{}
}
\author[Dominik Harmim \& Vojtěch Hertl]
{
	\texorpdfstring{
		\begin{columns}
			\column{.45\linewidth}
			\centering
			Dominik Harmim\\
			\texttt{xharmi00@stud.fit.vutbr.cz}
			\column{.45\linewidth}
			\centering
			Vojtěch Hertl\\
			\texttt{xhertl04@stud.fit.vutbr.cz}
		\end{columns}
		\vspace{1cm}
	}{}
}
\date{\today}
\institute
{
	Vysoké učení technické v~Brně\\
	Fakulta informačních technologií

	\bigskip

	\scalebox{0.3}{\includegraphics{inc/FIT_barevne_CMYK_CZ.pdf}}
}

\begin{document}

\maketitle

\begin{frame}{Přehled}
	\setbeamertemplate{section in toc}[sections numbered]
	\tableofcontents[hideallsubsections]
\end{frame}

\section{Úvod}

\begin{frame}{Zadání}
	\begin{alertblock}{Vlastní zadání}
		\begin{itemize}
			\item Webová aplikace.
			\item Speciální typ TODO listu.
			\item Každý uživatel má několik TODO listů.
			\item Položky na každém TODO listu je možné definovat
				globálně administrátorem systému a~navíc si každý uživatel
				může na každý TODO list přidávat i~svoje vlastní položky.
		\end{itemize}
	\end{alertblock}
\end{frame}

\begin{frame}{Průzkum kontextu použití}
	\begin{itemize}
		\item Při průzkumu cílové skupiny byl použit online dotazník.
		\item Požadavky na produkt vyplynuly z~výsledků dotazníku
			(jednoduchost, intuitivnost, jazyk aplikace, mobilní zařízení).
	\end{itemize}
\end{frame}

\section{Návrh a~testování}

\begin{frame}{Návrh UI}
	\begin{itemize}
		\item Sdílení nápadů v~týmu.
		\item Klíčové prvky:
			\begin{itemize}
				\item Výpis všech TODO listů daného uživatele.
				\item Práce s~TODO listem
				\item Definice položek TODO listu.
			\end{itemize}
	\end{itemize}
\end{frame}

\begin{frame}{Prototyp a~uživatelské testování}
	\begin{itemize}
		\item Společné vytvoření testovacího protokolu.
		\item Průběh testování.
		\item Výsledky a~závěry.
	\end{itemize}
\end{frame}

\section{Implementace}

\begin{frame}{Back-end implementace}
	\begin{itemize}
		\item Implementace v~jazyce PHP za použití Nette Framework.
		\item Použití MySQL databáze.
	\end{itemize}
\end{frame}

\begin{frame}{Front-end implementace}
	\begin{itemize}
		\item HTML\,--\,vytvoření obsahu stránek.
		\item CSS\,--\,responsivní stylování HTML.
		\item JavaScript\,--\,dynamický obsah a~validace.
		\item Použití knihovny Bootstrap.
		\item Demo Dominik Harmim
			\texttt{http://vut-itu-project.harmim.cz}.
		\item Demo Vojtěch Hertl
			\texttt{http://vut-itu-project-hertl.harmim.cz}.
	\end{itemize}
\end{frame}

\end{document}
